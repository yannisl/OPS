
\chapter{Introduction}

This report outlines a strategy for the successful Commissioning of the MEP Systems serving the Phase 3a and 3b City Center Project. The report examines the various systems and sub-systems related to the MEP services from the point of view of the following:

\begin{itemize}
\item Minimizing the period required for Commissioning.
\item Ensuring reliability of results.
\item Co-ordinating the work and allocation of responsibilities throughout all the Construction Team and the Professional team.
\item Commissioning Team Organization.
\end{itemize}

Currently  the Construction stage is at an advanced stage, with the possibility of starting up the commissioning activities imminent.

Concerns were raised by the Engineer and Project Manager, regarding issues that were experienced in Phase II of the Project, with the view of avoiding the unecessary delays experienced. This report examines some of the major issues.

The Commissioning Strategy forms part of an overall strategy for completing the Project that was developed earlier on. The Construction Strategy, outlined the following broad targets:

\begin{itemize}
\item Secure the wild air on dates
\item Completion of Towers
\item Power on
\item Qatar Cool
\end{itemize}

The Wild Air dates were achieved, as well as the Completion of MEP services on Towers for ceiling closures. Kahraama inspections took place and Rotana and Shangri-la power-on is expected soon. Merweb will follow. Once power-on is achieved the process for the Qatar Cool connection will start.


We still envisage adherence to this basic sequence, during the commissioning of the individual systems i.e., Towers first and then Podia.

The thrust of our strategy, focuses on what we term \textit{Progressive Commissioning}, which aims at \textit{early startup} of systems and the achievement of commissioning in a manner that progresses from smaller sections of the works to larger the systems. This enables the critical path to be formed by smaller duration activities, thus if a delay occurs in one area, it does not substantially add to the overall critical path duration. Early start-up has the advantage of making latent defects appear earlier, so that they can be remedied quickly. The experience from Phase II, is that two latent defects had an impact on the \textit{handover} of the building. A defect in a fitting that connects the chilled water pipes to the fan-coil-units and a factory defect on PRV valves supplied for the Fire Protection system. None of these defects could have manifested themeselves earlier. The PRV valves, were supplied with certification attesting to their functionality according to international standards and were duly submitted for approval. However, during prolonged operation they failed and are being replaced. Another issue that arose was that numerous fans underperformed.

These items are currently managed in Phase 3a \&3b. The chilled water system on the Towers has now been operated for more than 5 months and no such defects are evident. The PRV valves are being replaced and we expect changing all of them. Fan static pressures have been checked where possible and action is to be taken on the ECU drive units that indicate problem areas.

Another major impact on delays in Phase II, was an inordinate number of Engineer's Instructions received \textit{after} the start of the Commissioning of systems.
Although this is out of the Contractor's hands to an extend it can be managed, if all parties involved co-operate to minimize the impact of these instructions and the acceptance of the principles involved in \textit{Progressive Commissioning}, \ie\ not allowing these instructions to affect the overall commissioning progress. 


Other methods for speeding up exist. For example flushing can be speeded up and the water quality of a system improve dramatically with the introductio of sidestream filtration. This has already been pro-actively installed and contributed to wild-air being provided timeously. HS has suggested that these be incorprated permanently into the works to enable flexibilty during the commissioning and maintenance stages for further changes. Unfortunately the Engineer declined the suggestion and these will be removed, with consequent delays, if further changes to the chilled water system are instructed.

One last statement needs to be made. For commissioning to start, construction needs to be completed, and for construction to be completed the design must be completed. So far - and we are not endeavouring to make a contractual point here - this has not been the case. Construction has been delayed via the issuance of numerous EIs and other changes and the experience in Phase II points to this issue continuing well into the Construction period. 

 

\chapter{Construction Closure}

As stated in the Preface, for Commissioning to start, Construction needs to be completed, and for construction to be completed the Client's requirements and the Design needs to be completed. Tremendous effort has been put by all parties to accelerate the works where required to achieve Construction completion. Construction in many respects resembles traffic and traffic jams. Each and every individual that works for the Project - especially decision makers - can inadvertantly delay a Project. 

\section{Balance Works}

The balance of the works are currently managed by a full Construction Team and specialist `closure' teams. These closure teams are expected to stay on during the commissioning period to continue with snagging and minor changes.


\section*{Compartmentalization}

We suggest as part of the Closure of the Project, compartmentalization of areas and systems. For example, with the start of the Commissioning for Rotana, any further changes should be frozen. For example, before the Commissioning of the Chilled water system in Rotana the drawings for the Tower need to be upgraded to as-fitted status and no further changes allowed. In parallel the Chilled water system as shown on schematics would be frozen for Levels 8-47. We will be using Rotana as an example from now on onwards for convenience, but the arguments are equally applicable to the other Hotels and the Podia.

Once this is secure, it opens up the possibility for the immediate start of commissioning for the following services\sidenote{Upon power-on.}:
\begin{table}[htbp]
\begin{tabular}{lp{2in}}
\toprule
Service & Description\\
\midrule
Staircase Pressurization & Testing and balancing to commence at the earliest\\
fan-coil units           & Air-flow measurements\\
fan-coil units           & Water-flow balancing\\
smoke extract            & Testing and balancing\\
fresh-air system         & testing and balancing\\
Power-on                 & Power-on of all associated systems\\
Hot \& Cold Water        & Testing and balancing\\
Sprinkler system         & Testing\\
~				    &	 Setting of PRVs\\
\bottomrule
\end{tabular}
\end{table}
\bigskip

One issue that needs to be discussed is the activities of the Fit-out Contractor, which are expected to delay the Project\sidenote{Based on current mobilization plans, lack of any programme and the experience in Phase II.} Due to this we expect to progressively commission the fan-coil units and seek acceptance from the Engineer. As the Fit-out Contractor completes a room, we will endeavour that all systems that can possibly be commissioned to follow the Fit-out Contractor's progress.




\section{Commissioning Strategy}

\section{Team Organization}
The Team organization is shown in the organization chart accompanying this report.
\section{Programmes and planning}

\subsection{Commissioning Programme}

The commissioning programme, will initially be developed on a system by system basis, and be generated by HS Commissioning Management. Following an internal review with the Main Contractor planning department and Technical Managers, the programme will be issued to the Trade Contractors, for their comments, and acceptance with regards to durations only. Following to Trade Contractors acceptance, the programmes will be issued to the Professional Team of the \project\ Project.

The starting points for commissioning activities, will generally be driven by $2^{nd}$ or Final fix activities i.e pipework/ductwork/cable completion. This milestone will be the activity point to commence commissioning.

\subsection{Programme Reporting (Services Trades and Core Group}

Following commencement of the contractors commissioning HS will issue their programmes every two weeks at the weekly commissioning meeting and, indicating progress and reporting any anticipated delays and problems. Any delay must be accompanied with the Contractor's recovery plan, where possible.

It is the responsibility of HS to check the information with site surveys and report the effects to completion dates.

\section{Installation \& Inspection Monitoring}

\section{Factory Testing}
Completed

\section{Local Authorities Testing}
To follow as required (Kahraama inspections for two of the three substations has already been concluded successfully).

\section{Commissioning Monitoring}

HS Commissioning Management shall monitor the progress, of all MEP services packages. In order to undertake the task, it is essential, that all control devices (valves \& dampers), point schedules, cable schedules etc. are available. This information will utilized to develop a progress database. The summation of these points will be developed into a 'Master Summary' schedule.

The schedule will be updated, as and when tests are completed, and the progress recorded, into the Commissioning Programme.

\section{Witnessing}

To undertake witnessing of a commissioning activity, HOK will be ..

\subsection*{Client Witnessing Authority}

It will be the responsibility of the HS Commissioning Management to notify the witnessing authority - where possible of proposed tests. HEE will develop and manage a 'Master Statutory Authority Witness Schedule'.

\subsection*{Documentation}

HS will be reposnible to ensure that any defects noted at the time of inspection are rectified, prior to offering any system for witnessing. The documentation required will be:

\begin{itemize}
\item General arrangement/panel drawing/schematic of the system under test
\item Signed off inspection sheets.
\item Test equipment calibration sheet.
\item Test documentation and other similar documents.
\end{itemize}


\subsection{Percentage of Witnessing}

All tests will be witnessed by the Client's Representative at a minium of 25\%.
The exceptions to the 25\% witnessing are life safety and business critical systems where 100\% witnessing is required, or where test failures occur. 100\% witnessing will be required.

The following guide are the witnessing requirements.


\label{chap:listofservices}

\begin{longtable}{llll}
%\begin{tabular}{llll}
\toprule
Ref	&Package	&Commiss.	&Percentage \\
     &         &Entity    &witnessing\\
\midrule
1.00	&Mechanical	&	&\\
1.01	&Chilled Water Systems	&CML	&25\%\\
1.02	&Ductwork  &CML		&25\%\\
1.03  &Kitchen Extract System &CML &25\%\\
1.04	&Car Park Ventilation 	&CML &25\%\\
1.05	&Gas Fired steam Boilers &Specialist&25\%\\	
1.06	&Condenser Water System 	&CML&25\%     \\
1.07    &Cooling Tower &Specialist&25\%  \\
\midrule
2.00	&Electrical		&&\\
2.01	&Standby Generators	&Specialist&100\%\\	
2.02	&Medium Voltage System	&HS&25\%\\	
2.03	&Low Voltage System		&HS&25\%\\
2.04	&Earthing \& Lightning Protection &HS&100\%\\		
2.05	&Lighting \& Emergency Lighting  &HS&100\%\\		
2.06	&Aircraft Warning System &HS&100\%\\		
2.07	&UPS System	&Specialist&100\%\\	
2.08 &Power Factor  &Specialist &100\%\\
\midrule
3.00	&Public Health		&&\\
3.01	&Potable Water	&HS&25\%\\	
3.02	&Above Ground Drainage &HS&25\%\\		
3.03	&Below Ground Drainage  &HS&25\%\\		
3.04	&Water Features	     &Specialist&100\%\\	
3.05	&Gas Supply Installations	&Gasco&100\%\\	
3.06 &Gray Water Treatment Plant&Metito&100\%\\
\midrule		
4.00	&Fire Defense &Nafco&100\% \\	
4.01	&FHC, FH and Sprinkler Installations	&Nafco&100\%\\	
4.02	&Smoke Exhaust		&HS/CML&100\% \\
4.03	&Staircase Pressurization          &HS/CML&100\%\\		
4.04	&Fire Alarm		&Specialist&100\%\\
4.05	&FM-200		&Nafco&100\% \\
4.06	&Heliport Foam System	 &Nafco&100\% \\	
\midrule
5.00	&Specialist Services		&&\\
5.01	&Building Management System	&Shajan/HS&25\%\\	
5.02	&Security Access Control \& CCTV	&Specialist/HS&100\%\\	
5.03	&CO Monitoring		&Specialist&100\%\\
5.04	&Car Calling System		&Specialist&100\%\\
5.05	&Lighting Control		&HS/Specilaist&25\%\\
5.06	&Structured Cabling		&Specialist&25\%\\
5.07	&PA \& Background Music System &Specialist&25\%\\		
5.08	&IPTV \& Satellite System	&Specialist&100\%\\	
5.09	&Room Management System	&Specialist/HS&25\%\\
5.10 &Intercom                 &Specilaist/HS&25\%\\
5.11 &Audiovisual              &Specialist/HS&25\%\\
5.12 &Telephone                &Specialist/HS&25\%\\
5.13 &Wi-fi                    &Specialist/HS&25\%\\
\midrule
6.00 &Other                    &			&\\
6.01 & BMU                     &Specialist/HEE &25\%\\
6.02 & Swimming Pool           &Specialist     &25\%\\
6.03 &Irrigation System        &Specialist     &25\%\\
\bottomrule
%\end{tabular}			
\end{longtable}



\chapter{Air System Strategy}

\subsection*{A – Ductwork Installation (Al Habtoor-Specon)}

The ductwork contractor will install their ductwork, in line with the construction programme. Upon completion of their installation, it will be inspected internally, by the ductwork contractor. Any defect / non compliance shall then be rectified. Only then, will the contractor offer the system to HOK / EHAF for inspection.

The documentation issued for installation inspection will at a minimum include, the contractors own defect log. All of which, must be signed, by the ductwork contractor.

The request for witness will be undertaken utilizing the HS Commissioning Management Proforma, the following documentation shall be required, as a minimum:

\begin{itemize}

\item 	Ductwork General Arrangement, with the ductwork, clearly highlighted.

\end{itemize}

Following acceptance of the ductwork inspection, the system will be released for lagging. Upon completion of the lagging, it will be inspected by the ductwork contractor, and offered to the team for inspection.

\subsection{B – Ductwork System Static Completion (Al Habtoor-Specon)}

Static completion involves checking that the system as a whole is complete, safe, and ready to energize the fan associated with the system. The static completion checks are as stated in CIBSE Commissioning Code A, Section A1.2 to A1.6, AABC National Standard Code, ASHRAE, DW 143, but are generally as follows:

\begin{itemize}
\item 	Check the ductwork has been signed off in respect to the installation and the ductwork has been pressure tested.
\item 	Check that the ductwork is clean, system components are installed, and the system is safe to energize the fan.
\item 	Check that all components are installed and operate e.g. fire dampers, volume control dampers, motorized dampers, diffusers and filters etc.
\item	Check that the fan is safe to operate e.g. the impeller is free to rotate, anti vibration mounts removed, etc.
\end{itemize}




\chapter{Electrical Commissioning Strategy}

\subsection{Electrical Installation (HS)}
\label{sub:electrical}
The Electrical Contractor (HS), will intall their containment, in line with the construction programme. Upon completion of the installation, it will be inspected internally, by the Electrical Contractor and HEE. Any defect/non-compliance shall then be rectified. Only then, the Contractor will offer the system to EHAF/HOk for inspection. The documentation issued for installation inspection will include panel/drawings/schematics/containement drawings. All of which must be signed by the Electrical Contractor.

\subsection{Cable Pulling \& Terminations (HS)}

Following acceptance of the installation by HOK, the Electrical Contractor, will commence the cable insatllation. Upon completion of the cable installation, the Electrical Contractor will again inspect all cables installed. Any defect/non compliance shall then be rectified. The same process as stated within subsection \ref{sub:electrical} will be required.\index{Cable pulling}

\subsection{Static Completion (HS)}
\index{static completion}\label{sec:static}

Following sign-offs, the system will be deemed that is phusically completed (static completion). Static completion involves checking that the systems as a whole are complete, safe and ready for setting to work. The static completion checks are generally as follows:

\begin{itemize}
\item Check the electrical containment has been signed off.
\item Check all cables and terminations have been signed off.
\item Check the system for cleanliness, to facilitate test, and energise.
\item Check that all system components are installed and connected.
\end{itemize}

Following static completion checks the Electrical Contractor will issue a statement or their own pre-start procedures to HS Commissioning Management. HOK \& EHAF who will then inspect the installation to confirm static completion has been achieved.

\subsection{Dead Testing of Cables (HS)}

\subsection{Power Distribution \& LIve Testing of Cables (HS)}


\chapter{Hydraulic System Commissioning Strategy}

\section{Overview}
The hydraulic system comprises the Chilled water System, the Condenser water System etc.
\section{Testing and Balancing Contractor (CML)}
A Specialist Independent Contractor has been assigned to carry out these works (CML). 

\section{Pipework Installation}

\section{Pre-commissioning}

\section{Mechanical equipment}

\section{Mechanical Power Wiring Dead testing of cables (HS)}

\section{BMS Installation (Shahjan Trading)}

\section{BMS Point to Point Testing (Shajan Trading)}

\section{Mechanical Commissioning}

Following satisfactory sign-offs, the BMS Contractor will make a formal application, to energise the DCC. This will be undertaken via a "Power On Protocol". The Mechanical Contractor will then carry out the commissioning of the Mechanical Plant, which will generally involve the following:

\begin{enumerate}
\item Start up of pump(s).
\item Setting overloads and timers, checking fuses.
\item Simulation of interfaces (\eg Fire and BMS).
\item Stroking of motorised valves and checking auxiliary switches.
\item Commissioning VFD drives
\item Leaving test links in place to allow follow on trades to carry out commissioning.
\item Liaise with the BMS Contractor to ensure all motorised valves operate correctly.
\end{enumerate}

Where Mechanical commissioning is deemed to have passed, the Mechanical contractor will submit their test documentation and request formal demonstration to the Engineer. The documentation issued, will be a minimum include Mechanical Commissioning Checklist signed off by the BMS contractor.

\section{Mechanical Equipment Live Testing of Cables (HS)}

Following satisfactory sign off of the Mechanical equipment commissioning, the electrical contractor will carry out live testing of the power cabling in accordance with BS7671. Where live testing is deemed to have passed the Electrical Contractor will submit the test documentation and offer the installation for witnessing by the Engineer. The request for witness, will be undertaken utilizing the HS Commissioning Management Proforma, the following documentation shall be required as a minimum.

\section{Setting to work}

Following the setting to work, the Mechanical Contractor will issue the setting to work checklist and equipment checklist to EHAF/HOK who will then inspect, to confirm the setting to work has been achieved.

The documentation issued for the setting to work will at a minimum include a setting to work checklist a system schematic, a pump performance sheet(s) and calibrartion certificates.

\section{Testing and Balancing}

Following acceptance of the completion certificate the Testing \& Balancing contractor will carry out commissioning of the water system which generally involve terminal balancing, pump performance as stated in CIBSE Commissioning Code W, section W7, will submit the test documentation and offer to witness to HOK. The request for witness will be undertaken utilizing the HS Commissioning Management Proforma. The following documentation shall be required as a minimum:

\begin{itemize}
 \item Schematic/General arrangement for area under test
 \item Inspection Sign Off Sheets
 \item Calibrartion Certificate for Instrumentation
 \item Test documentation, with the Test engineer's signature, \ie\ Pump test sheet, proportional balancing sheets, overload settings, pump curves etc.
\end{itemize}

\section{BMS Commissioning (Shajan Trading)}

Following satisfactory sign off of the Testing and Balancing the BMS contractor will carruy out Phase 2 of the pre-commissioning works as stated in CIBSE Commissioning Code C, sections C5.6 to C5.8 and generally as follows:

\begin{itemize}
\item Sensor calibration.
\item Motorized valve drive and stroking.
\item Check input and output signal.
\end{itemize}

The point to graphic testing will be undertaken from the field point to the headed graphic.\index{BMS!graphic}\index{BMS!chilled water}

\section{Commissioning}

Upon satisfactory completion of the BMS point to graphic testing the Qatar Cool specialist will carry out commissioning of the main plant in Basement 3. It is critical that the MEchanical Contractor and the BMS Contractor provide attendances as required to carry out joint commissioning especially when integrating with Qatar Cool PLC.

\section{BMS Commissioning}


\textbf{Note: Connections to Qatar Cool (HS)}

Prior to making permannet connections to ``Qatar Cool'' chilled water system, it is essential that the Mechanical Contractor validate the system parameters. this should be undertaken by the Mechanical Commissioning Contractor, and also validated the the BMS headed and Qatar Cool PLC.



\chapter{Integrated System Test}
HOK will generate an integrated systems testing document based on the requirements of the contractor specifications. It will be the responsibility of HOK to manage the testing and the sign off process.


\chapter{O\&M Manuals}

HOK will review and input their comments. HS Commissioning Management will track all test and Commissioning Test Documents.

O \&M Manuals are to be prepared by HS in conjuction with the various service providers. A special section of the Commissioning Team will do this.

\chapter{Training}

Training will be undertaken in line with Section 15990, clause 5.3.1.2.5. HOK will review and input the Client requirements. HS will track all training requirements.


\chapter{Conclusions}

This report has outlined the Organization of the Commissioning Team and has broadly mapped the Methodology and the Strategy to be employed during the Commissioning period. It discussed, how to pro-actively minimize the potential for delays. It did not discuss in detail all the services, as these will be covered by individual Method Statements.




































