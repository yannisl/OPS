
\begin{fullwidth}
\begin{figure*}[htbp]
\newcommand{\gbar}[3]{\ganttbar[crosshatch, color=orange]{#1}{#2}{#3}}

\definecolor{lightgray}{rgb}{0.7,0.7,0.7}
%% #1 11 rows deep
%% #2 24 cells wide

%% Define short-cuts for months
%%


\def\gantscale#1{}

\newcounter{acount}
\setcounter{acount}{0}
\stepcounter{acount}
\addtocounter{acount}{10}



%\ifnum\theacount>10 number is greater than ten \else number is less than ten\fi


%% Define the months just in case they have not been set
%% before. This needs to be internationalized.
\def\mon#1{\ifcase#1\or%
  Jan\or Feb\or Mar\or Apr\or May\or Jun\or%
  Jul\or Aug\or Sept\or Oct\or Nov\or Dec\else Error 
\fi}


%% a helper macro to add a comma and a space to make the
%% code later on a bit more cleaner

\def\addcomma{, }

%% We define a new counter for looping through the months
%% in order to create the calendar on top
%% this will need to be expanded later 
%\newcounter{monthcount}
%\setcounter{monthcount}{1}
%\whiledo{\value{monthcount}<12}
%  {%
%   \mon{\themonthcount}%
   % adds comma except last one
 %  \ifnum\themonthcount<11 \addcomma\else\relax\fi %
   
  % \stepcounter{monthcount}% 
  %}


%% prints date
%\mon{\numexpr 1+3\relax} \intcalcAdd{\number\year}{1} \relax

%% Global setup
%% 

\xdef\monthstoplot{48}
\xdef\linesdeep{17}

%% Chart drawn in the margins
%% This is the main routine for the chart

%% We define a scale
%
\def\ganttscale#1{#1}



\hskip-2.5cm\begin{gantt}[xunitlength=\ganttscale{0.25cm},  fontsize=\footnotesize, drawledgerline=false]{\linesdeep}{\monthstoplot}

%% Define the title of the chart
%%
%% 
\newcommand\GanttTitle[1]{
\ganttitle
    \titleelement{\textsc{#1}}{\monthstoplot}
\endganttitle
}


\GanttTitle{ECU and Black Steel Installation - Rotana}

\begin{ganttitle}
%% We define a new counter for looping through the months
%% in order to create the calendar on top
%% this will need to be expanded later 
\newcounter{monthcount}
\setcounter{monthcount}{1}
\whiledo{\value{monthcount}<13}
  {%
   \mon{\themonthcount}%
   % adds comma except last one
   %\ifnum\themonthcount<11 \addcomma\else\relax\fi %
   \titleelement{\mon{\themonthcount}}{4}
   \stepcounter{monthcount}% 
  }
 
\end{ganttitle}

%% We now loop to typeset the week labels
%% this needs to become more sophisticated
%% as some months do not have 4 weeks
\begin{ganttitle}
   \newcounter{weekcount}
   \setcounter{weekcount}{1}
   \whiledo{\value{weekcount}<13}
  {%
   \mon{\theweekcount}%
   \numtitle{1}{1}{4}{1}
   \stepcounter{weekcount}% 
  }      
\end{ganttitle}
 
%% We define some new colors
\definecolor{red}{rgb}{1,0,0}
% activities are plotted here
\xdef\prop{crosshatch, color=orange}

\ganttbar[crosshatch, color=green]{GR-RO-ECU-1 - horizontal}{0}{8}
\ganttbar[crosshatch, color=DarkGreen]{GR-RO-ECU-1 - vertical}{0}{8}

\ganttbar[crosshatch, color=green]{L1-RO-ECU-1 - horizontal}{8}{6}
\ganttbar[crosshatch, color=purple]{L1-RO-ECU-1 - vertical}{7}{6}

\ganttbar[crosshatch, color=blue]{L1-RO-ECU-2 - horizontal}{0}{6}
%% Level 6
\ganttbar[dots, color=violet]{L6-RO-ECU-1 - horizontal}{5}{4}
\ganttbar[dots, color=purple]{L6-RO-ECU-1 - vertical}{0}{8}

\ganttbar[crosshatch, color=blue]{L6-RO-ECU-2 (horizontal)}{8}{6}
\ganttbar[crosshatch, color=violet]{L6-RO-ECU-2 (vertical)}{8}{6}


\ganttbar[dots, color=orange]{L6-RO-ECU-4 (horizontal)}{5}{6}
\ganttbar[dots, color=DarkGreen]{L6-RO-ECU-4 (vertical)}{7}{6}


\ganttbar[crosshatch, color=purple]{L7-RO-ECU-1}{4}{3}

\ganttbar[crosshatch, color=brown]{Commissioning}{16}{6}

\end{gantt}

\colorbox{green}{\textcolor{black}{Team-1}} Currently working in B1 (Andreas)\\[3pt]
\colorbox{DarkGreen}{\textcolor{white}{Team-2}} Currently working in L6 shafts going down (Fitos)\\[3pt]
\colorbox{purple}{\textcolor{white}{Team-3}} Currently working in L6 shafts going down (Fitos)\\[3pt]
\colorbox{blue}{\textcolor{white}{Team-4}} Currently working in B1 to be moved to L1 (Andreas)\\[3pt]
\colorbox{violet}{\textcolor{white}{Team-5}} New Team HS, (Fitos)\\[3pt]
\colorbox{orange}{\textcolor{white}{Team-6}} New Team HS, (Andreas)\\[3pt]
\caption{Rotana installation and commissioning program.}
\label{plan}
\end{figure*}

\end{fullwidth}



%   
%\section{The gantt package}
%
%In the following you will find a short description of environments and commands: 
%
%The gantt environment draws the canvas of a gantt figure (realized as tikzpicture)
%The usage is |begin{gantt}[...]{no of Tasks to plot}{no of time slots}|
%The optional argument |[...]| can be filled in a |key=value| syntax, using one or more of the following keys:
%
%\begin{description}
%\item [xunitlength]  length of one time slot (default: 1 cm)
%\item [fontsize] fontsize of labels (default: |\normalsize|)
%\item [titlefontsize] fontsize of title section (default: |\small|)
%\item [drawledgerline] Switch to enable/disable the drawing of horizontal ledger lines (default value: false)
%\item [ganttitle] is the environment for drawing the title section
%\item [titleelement] draws one element of the title
%usage: |\titleelement{label}{length}}|
%\end{description}
%
%
%
%
%The \cmd{numtitle} draws a numbered sequence of title elements:
%usage: 
%
%\begin{teX}
%\numtitle{start number}{increment}{end number}{length of each title element}
%\end{teX}
%
%\cmd{ganttbar} draws a single, unconnected bar for representing a task
%usage: 
%
%\begin{teX}
%\ganttbar[pattern]{label}{start}{length}
%\end{teX}
%
%where the optional pattern argument is a tikz pattern, nice patterns for tasks are: 
%north west lines (default), north east lines, crosshatch, crosshatch dots, grid, ...
%
%
%\begin{verbatim}
%\cmd{ganttcon} draws an arrow between to bars with specified coordinates
%usage: \ganttbar{startx}{starty}{endx}{endy}
%
%\ganttbarcon draws a single bar \textit{and} connects the bar with the previous bar for consecutive tasks
%usage: 
%
%\begin{teX}
%\ganttbar[pattern]{label}{start}{length}
%\end{teX}
%
%where the optional pattern argument is a tikz pattern, nice patterns for tasks are: north west lines (default), north east lines, crosshatch, crosshatch dots, grid.
%
%\ganttgroup draws a bar to group tasks
%usage: \ganttgroup{label}{start}{length}
%
%\end{verbatim}
%\section{Physical completion}

%\label{g:test}
%
%\numberLineAt{10}
%\begin{teX}
%\def\mon#1{\ifcase#1\or%
%  Jan\or Feb\or Mar\or Apr\or May\or Jun\or%
%  Jul\or Aug\or Sept\or Oct\or Nov\or Dec\else Error 
%\fi}
%\end{teX}











%These are notes for the gantt chart
