\documentclass[a4paper, oneside, justified=true, sfsidenotes]{tufte-book} % for a long document

\hypersetup{colorlinks}% uncomment this line if you prefer colored hyperlinks (e.g., for onscreen viewing)

%%
% Book metadata
\title{CONSTRUCTION\\{\noindent{STRATEGY}}FOR MEP{}}
\author[Dr Y Lazarides]{City Centre Phase 3a \& 3b}
\publisher{HEE-Specon}

%%
% If they're installed, use Bergamo and Chantilly from www.fontsite.com.
% They're clones of Bembo and Gill Sans, respectively.
\IfFileExists{bergamo.sty}{\usepackage[osf]{bergamo}}{}% Bembo
\IfFileExists{chantill.sty}{\usepackage{chantill}}{}% Gill Sans
\usepackage{siunitx}
\usepackage{multido}    % for multido looping
\usepackage{colortbl}
\usepackage{microtype}
\usepackage{cases}
\usepackage{pgfplots}
\usepackage{pgffor}
\usepackage{lettrine}
\usepackage{helvet}
\usepackage{beramono}
\usepackage{soul}
\usepackage{listings}
\usepackage{calc}
\usepackage{multicol}
\usepackage{epigraph}

% uncomment to have draft on pages printed
\usepackage[some]{background}
%%
% Just some sample text
\usepackage{lipsum}
\usepackage{pstricks}
\usepackage{pst-plot}

%%
% For nicely typeset tabular material
\usepackage{booktabs}

%%
% For graphics / images
\usepackage{graphicx}
\setkeys{Gin}{width=\linewidth,totalheight=\textheight,keepaspectratio}
\graphicspath{{graphics/}}

% The fancyvrb package lets us customize the formatting of verbatim
% environments.  We use a slightly smaller font.
\usepackage{fancyvrb}
\fvset{fontsize=\normalsize}

%%
% Prints argument within hanging parentheses (i.e., parentheses that take
% up no horizontal space).  Useful in tabular environments.
\newcommand{\hangp}[1]{\makebox[0pt][r]{(}#1\makebox[0pt][l]{)}}

%%
% Prints an asterisk that takes up no horizontal space.
% Useful in tabular environments.
\newcommand{\hangstar}{\makebox[0pt][l]{*}}

%%
% Prints a trailing space in a smart way.
\usepackage{xspace}




% Prints the month name (e.g., January) and the year (e.g., 2008)
\newcommand{\monthyear}{%
  \ifcase\month\or January\or February\or March\or April\or May\or June\or
  July\or August\or September\or October\or November\or
  December\fi\space\number\year
}



% Inserts a blank page
\newcommand{\blankpage}{\newpage\hbox{}\thispagestyle{empty}\newpage}



% Typesets the font size, leading, and measure in the form of 10/12x26 pc.
\newcommand{\measure}[3]{#1/#2$\times$\unit[#3]{pc}}

% Macros for typesetting the documentation
\newcommand{\hlred}[1]{\textcolor{Maroon}{#1}}                            % prints in red
\newcommand{\hangleft}[1]{\makebox[0pt][r]{#1}}
\newcommand{\hairsp}{\hspace{1pt}}% hair space
\newcommand{\hquad}{\hskip0.5em\relax}% half quad space
\newcommand{\TODO}{\textcolor{red}{\bf TODO!}\xspace}
\newcommand{\ie}{\textit{i.\hairsp{}e.}\xspace}
\newcommand{\eg}{\textit{e.\hairsp{}g.}\xspace}
\newcommand{\na}{\quad--}% used in tables for N/A cells
\providecommand{\XeLaTeX}{X\lower.5ex\hbox{\kern-0.15em\reflectbox{E}}\kern-0.1em\LaTeX}
\newcommand{\tXeLaTeX}{\XeLaTeX\index{XeLaTeX@\protect\XeLaTeX}}
% \index{\texttt{\textbackslash xyz}@\hangleft{\texttt{\textbackslash}}\texttt{xyz}}
\newcommand{\tuftebs}{\symbol{'134}}% a backslash in tt type in OT1/T1
\newcommand{\doccmdnoindex}[2][]{\texttt{\tuftebs#2}}% command name -- adds backslash automatically (and doesn't add cmd to the index)
\newcommand{\doccmddef}[2][]{%
  \hlred{\texttt{\tuftebs#2}}\label{cmd:#2}%
  \ifthenelse{\isempty{#1}}%
    {% add the command to the index
      \index{#2 command@\protect\hangleft{\texttt{\tuftebs}}\texttt{#2}}% command name
    }%
    {% add the command and package to the index
      \index{#2 command@\protect\hangleft{\texttt{\tuftebs}}\texttt{#2} (\texttt{#1} package)}% command name
      \index{#1 package@\texttt{#1} package}\index{packages!#1@\texttt{#1}}% package name
    }%
}% command name -- adds backslash automatically
\newcommand{\doccmd}[2][]{%
  \texttt{\tuftebs#2}%
  \ifthenelse{\isempty{#1}}%
    {% add the command to the index
      \index{#2 command@\protect\hangleft{\texttt{\tuftebs}}\texttt{#2}}% command name
    }%
    {% add the command and package to the index
      \index{#2 command@\protect\hangleft{\texttt{\tuftebs}}\texttt{#2} (\texttt{#1} package)}% command name
      \index{#1 package@\texttt{#1} package}\index{packages!#1@\texttt{#1}}% package name
    }%
}% command name -- adds backslash automatically
\newcommand{\docopt}[1]{\ensuremath{\langle}\textrm{\textit{#1}}\ensuremath{\rangle}}% optional command argument
\newcommand{\docarg}[1]{\textrm{\textit{#1}}}% (required) command argument
\newenvironment{docspec}{\begin{quotation}\ttfamily\parskip0pt\parindent0pt\ignorespaces}{\end{quotation}}% command specification environment
\newcommand{\docenv}[1]{\texttt{#1}\index{#1 environment@\texttt{#1} environment}\index{environments!#1@\texttt{#1}}}% environment name
\newcommand{\docenvdef}[1]{\hlred{\texttt{#1}}\label{env:#1}\index{#1 environment@\texttt{#1} environment}\index{environments!#1@\texttt{#1}}}% environment name
\newcommand{\docpkg}[1]{\texttt{#1}\index{#1 package@\texttt{#1} package}\index{packages!#1@\texttt{#1}}}% package name
\newcommand{\doccls}[1]{\texttt{#1}}% document class name
\newcommand{\docclsopt}[1]{\texttt{#1}\index{#1 class option@\texttt{#1} class option}\index{class options!#1@\texttt{#1}}}% document class option name
\newcommand{\docclsoptdef}[1]{\hlred{\texttt{#1}}\label{clsopt:#1}\index{#1 class option@\texttt{#1} class option}\index{class options!#1@\texttt{#1}}}% document class option name defined
\newcommand{\docmsg}[2]{\bigskip\begin{fullwidth}\noindent\ttfamily#1\end{fullwidth}\medskip\par\noindent#2}
\newcommand{\docfilehook}[2]{\texttt{#1}\index{file hooks!#2}\index{#1@\texttt{#1}}}
\newcommand{\doccounter}[1]{\texttt{#1}\index{#1 counter@\texttt{#1} counter}}

% Generates the index
\usepackage{makeidx}
\makeindex
%\setcounter{secnumdepth}{1}

% Some macros for drawing gantt diagrams using tkiz.
%
% Martin Kumm, 17.05.2010

\usepackage{tikz,pgffor,xkeyval,ifthen,calc,forloop}
\usetikzlibrary{patterns}
\usetikzlibrary{arrows}

\newcounter{ganttnum}
\newcounter{ganttwidth}

\newlength{\ganttlastx}
\setlength{\ganttlastx}{0cm}

\newlength{\gantttmpa}
\newlength{\gantttmpb}

\setlength{\unitlength}{1cm}

\newlength{\titleunitlength}
\setlength{\titleunitlength}{1cm}

\newcounter{gantttitlenum}

\makeatletter
\define@key{ganttx}{xunitlength}{%
  \setlength{\unitlength}{#1}
}
\define@boolkey{ganttx}{drawledgerline}{}

\define@key{ganttx}{fontsize}{\def\ganttfontsize{#1}}
\define@key{ganttx}{titlefontsize}{\def\gantttitlefontsize{#1}}

 \presetkeys{ganttx}{drawledgerline=false,xunitlength=1cm,titlefontsize=\small,fontsize=\normalsize}{}


% The gantt environment draws the canvas of a gantt figure (realized as tikzpicture)
% The usage is \begin{gantt}[...]{no of Tasks to plot}{no of time slots}
% The optional argument [...] can be filled in a key=value syntax, using one or more of the following keys:
%
% xunitlength - length of one time slot (default: 1 cm)
% fontsize - fontsize of labels (default: \normalsize)
% titlefontsize - fontsize of title section (default: \small)
% drawledgerline  - Switch to enable/disable the drawing of horizontal ledger lines (default value: false)

\newenvironment{gantt}[3][]{%
  \begin{tikzpicture}[draw=gray, yscale=.7,xscale=1]
  \tikzstyle{time}=[coordinate]
  \setkeys{ganttx}{#1}{%
      \setcounter{ganttwidth}{#3}

      \setcounter{ganttnum}{0}
      \newcount\ganttx
      \newcount\ganttheight
      \ganttheight=#2 %define the number of Y ticks
      \advance\ganttheight by 1

      \def\ganttxstringtop{};
      \def\ganttxstringbottom{};

      \ganttx=0
      \draw (0,0.5) node[above] {\ganttxstringtop} -- (0,1.4-\ganttheight)  node[below] {\ganttxstringbottom};
      \draw (\value{ganttwidth}*\unitlength,0.5) node[above] {\ganttxstringtop} -- (\value{ganttwidth}*\unitlength,1.4-\ganttheight)  node[below] {\ganttxstringbottom};

      %draw grid
      \foreach \t in {1,2,...,\value{ganttwidth}}{
        \ganttx=1
        \multiply\ganttx by \t

        \draw[very thin, color=gray] (\unitlength*\t,-0.5) node[above] {\ganttxstringtop} -- (\unitlength*\t,1.4-\ganttheight)  node[below] {\ganttxstringbottom};
      }
     %draw x axis
     \draw[very thin, color=gray] (0,-#2+0.4) -- (\value{ganttwidth}*\unitlength,-#2+0.4);
}
}{\end{tikzpicture}}


% ganttitle is the environment for drawing the title section
\newenvironment{ganttitle}[1][]{%
  \setlength{\ganttlastx}{0 cm}
}{%
  \setlength{\ganttlastx}{0 cm}
  \addtocounter{ganttnum}{-1}%
}

% \titleelement draws one element of the title
% usage: \titleelement{label}{length}
\newcommand{\titleelement}[2]{
  \setlength{\gantttmpa}{\unitlength*#2}
  \divide\gantttmpa by 2;

  \draw (\the\ganttlastx,\value{ganttnum}) rectangle (\the\ganttlastx+#2*\unitlength,\value{ganttnum}+0.6);
  \node at (\the\ganttlastx+\the\gantttmpa,\value{ganttnum}+0.25) {%
    \gantttitlefontsize #1%
  };

  \setlength{\ganttlastx}{\ganttlastx+\unitlength*\real{#2}}
}


% \numtitle draws a numbered sequence of title elements
% usage: \numtitle{start number}{increment}{end number}{length of each title element}
\newcommand{\numtitle}[4]{
  \forloop[#2]{gantttitlenum}{#1}{\(\value{gantttitlenum} < #3\) \or \(\value{gantttitlenum} = #3\)}%
  {%
    \titleelement{\thegantttitlenum}{#4}
  }
}


% \ganttbar draws a single, unconnected bar for representing a task
% usage: \ganttbar[pattern]{label}{start}{length}
% where the optional pattern argument is a tikz pattern, nice patterns for tasks are: 
% north west lines (default), north east lines, crosshatch, crosshatch dots, grid, ...
\newcommand{\ganttbar}[4][north west lines]{
  \setlength{\gantttmpa}{#3\unitlength}
  \setlength{\gantttmpb}{#4\unitlength}
  \setlength{\gantttmpb}{\gantttmpa+\gantttmpb}

  \ifKV@ganttx@drawledgerline
    \draw[color=gray][very thin] (0,\value{ganttnum}-0.2) -- (\value{ganttwidth}*\unitlength,\value{ganttnum}-0.2);
  \fi
  \node at (0,\value{ganttnum}) [anchor=base east] {%
		\ganttfontsize #2%
  };
  \draw[pattern=#1] (\gantttmpa,\value{ganttnum}+0.1) rectangle (\gantttmpb,\value{ganttnum}+0.5);
  \setlength{\ganttlastx}{\gantttmpb}
  \addtocounter{ganttnum}{-1}
}

% \ganttcon draws an arrow between to bars with specified coordinates
% usage: \ganttbar{startx}{starty}{endx}{endy}
\newcommand{\ganttcon}[4]{
    \draw[-latex,rounded corners=1px] (#1\unitlength,-#2+0.1 + 0.2) -- (#1\unitlength+0.4*\unitlength,-#2+0.1+0.2) -- (#1\unitlength+0.4*\unitlength,-#2-0.4+0.2) -- (#1\unitlength-0.4*\unitlength,-#2-0.4+0.2) -- (#1\unitlength-0.4*\unitlength,-#4+0.1+0.2) -- (#3\unitlength,-#4+0.1+0.2);
}


% \ganttbarcon draws a single bar *and* connects the bar with the previous bar for 
% consecutive tasks
% usage: \ganttbar[pattern]{label}{start}{length}
% where the optional pattern argument is a tikz pattern, nice patterns for tasks are: 
% north west lines (default), north east lines, crosshatch, crosshatch dots, grid, ...
\newcommand{\ganttbarcon}[4][north west lines]{
  \ifdim\the\ganttlastx=#3\unitlength
    \draw[-latex,rounded corners=1px] (\the\ganttlastx,\value{ganttnum}+1.1 + 0.2) -- (\the\ganttlastx+0.4*\unitlength,\value{ganttnum}+1.1+0.2) -- (\the\ganttlastx+0.4*\unitlength,\value{ganttnum}+0.6+0.2) -- (\the\ganttlastx-0.4*\unitlength,\value{ganttnum}+0.6+0.2) -- (\the\ganttlastx-0.4*\unitlength,\value{ganttnum}+0.1+0.2) -- (#3\unitlength,\value{ganttnum}+0.1+0.2);
  \else
    \draw[-latex,rounded corners=1px] (\the\ganttlastx,\value{ganttnum}+1.1 + 0.2) -- (\the\ganttlastx+0.4*\unitlength,\value{ganttnum}+1.1+0.2) -- (\the\ganttlastx+0.4*\unitlength,\value{ganttnum}+0.1+0.2) -- (#3\unitlength,\value{ganttnum}+0.1+0.2);
  \fi
  \ganttbar[#1]{#2}{#3}{#4}  
}


% \ganttgroup draws a bar to group tasks
% usage: \ganttgroup{label}{start}{length}
\newcommand{\ganttgroup}[3]{%
  \setlength{\gantttmpa}{#2\unitlength}
  \setlength{\gantttmpb}{#3\unitlength}
  \setlength{\gantttmpb}{\gantttmpa+\gantttmpb}

  \ifKV@ganttx@drawledgerline
    \draw[dotted] (0,\value{ganttnum}-0.2) -- (\value{ganttwidth}*\unitlength,\value{ganttnum}-0.2);
  \fi
  \node at (0,\value{ganttnum}) [anchor=base east] {%
    \ganttfontsize #1%
   };
  \fill[black] (\gantttmpa-0.14cm,\value{ganttnum}+0.2) rectangle (\gantttmpb+0.14cm,\value{ganttnum}+0.4);
  \draw[diamond-diamond] (\gantttmpa-0.14cm,\value{ganttnum}+0.2) -- (\gantttmpb+0.14cm,\value{ganttnum}+0.2);

  \setlength{\ganttlastx}{\gantttmpb}
  \addtocounter{ganttnum}{-1}
}


\makeatother
