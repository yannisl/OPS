
\cleardoublepage

\chapter*{Executive Summary}


This report outlines a strategy to be employed in order to complete this Project on time. The deadline of the {\protect\deadline}~is expected to be met, where no serious constraints exist. The report analyses the strategy to be employed, and sets milestones and expectations. 

It is possible for some of the systems to be commissioned by the deadline. However, proper commissioning and the completion of certain systems, will only be possible after this date. 

Full recovery with the inclusion of additional works being instituted is not possible. A suitable extension of time should be claimed, in order to safeguard HEE and Specon's entitlements and reputation, even if currently HOK is rejecting all claims. 

In Summary, Towers will be substantially completed with false ceiling clearances by mid September and Podia will follow with Testing and Commissioning which are inter-dependent with fit-out installations, and other contractors. Power on will be possible to achieve on Towers Shangri La and Rotana. Merweb will follow, owing to Commercial delays. It is imperative that we appoint an additional subcontractor immediately to help overcome this.
% Start the main matter (normal chapters)



\chapter{Introduction}
\label{ch:the-problem}

\newthought{Can this Project be completed by  the $30^{th}$  of  October 2010} or within a reasonable time afterwards? 
This report answers this question.

\newthought{Current MEP Resources} number over 3000 technicians and over 200 staff. From an MEP resource point of view, it is perhaps unique in Qatar and at first glance, completion should not present a problem. However, imbalances in trades as a result of weaknesses of deployed subcontractors and a general lack of skilled technicians in Doha and an ever increasing number of engineering and material issues endangers completion dates.


\newthought{Materials awaiting delivery} are approximately worth QR~37.5~million and approximately another 20 million has been budgeted for orders still to be placed. \sidenote{The amount is just an estimate and comprises items such as IPTV, which have not as yet been finalized by the Consultants.}

\newthought{A strategy of divide and conquer} will be implemented. By identifying areas where completion is not inhibited we can target ({\tt{SMART}}\sidenote{small, meanigfull, achievable, reasonable and timely}) 
sections of the works where effort can be focused.  Tight integration and adoption of this principle by all parties is essential for its success. \index{\texttt{SMART}}

\newthought{Progress is currently being measured weekly} and a target progress of 3\% per week is expected in order to ensure that completion is achieved by the \deadline. At best the measurement is a rough indicator as neither HEE's Planning Department nor Specon's has the methodology or systems to measure output as accurately as 1\%. However, as an indicator, especially for Towers it provides a metric that can track overall performance.


\section{Constraints}
The Site has been severely constrained by Engineering and Materials over the last few months and efforts to correct it have been paying off. For ductwork we are currently short of experienced technicians. Numerous Engineering constraints still remain and these are discussed in detail in the sections that follow.

The situation is partially recoverable, but a concerted effort must be made to 
properly document delays that are caused by other parties in order to safeguard HEE and Specon's interests and reputation. 

\section{Approach}

The finalization of the Project is to be approached from changing-over to a {\em Systems} type of thinking. \sidenote{There are over 40 different systems per Hotel to be completed and  commissioned. A full list is provided in {\em Appendix\/ A}.}  By ensuring that a single person is responsible for a system - and the target is simply - to activate this system, a systematic approach of removing obstacles is possible. 

Although, from a Main Contractor's point of view emphasis is placed in completing areas, from an MEP point of view, priority must be given to completing systems. 

Early activation of systems is of benefit to both parties, as unforseen problems surface early,  thus avoiding re-opening of ceilings or walls and redundant works. It also opens the possibility of progressive hand-over.  By adopting such an approach the whole Team focuses towards common goals and completion.

The systems approach does not mean that one cannot focus on areas. By carefull planning of completion dates, both systems and areas can be completed, but in such a way that neither the main contractor nor the MEP subcontractor is delayed. (See for example  the section on the  Podia on page  \pageref{sec:podium}).

Logistics, especially vertical transportation of personnel and goods needs to be addressed, this is discussed in detail on page \pageref{sec:logistics}, dealing in detail with Logistics, Materials and Stores.\index{Logistics!logistics}
