\chapter{Business edge}

\section*{Introduction}

In order to build a business edge we need to define the business better, decide on the sector we are operating in and build an edge that will differentiate us from our competitors.

\subsection{What are we offering}
Are we selling materials, which we incidentally install? Are we selling services, integrate Suppliers and Subcontractors and execute the works? What is our \textbf{product}?

Our product should be focused well defined and enable us to sell it in a way we differentiate from the opposition.
Our main opposition can be divided into three main groups:

\begin{itemize}
\item Low cost Companies, mostly from India suh as ETA and Voltas. Some of these Companies as for example Voltas, are backed by big groups operating in many sectors. Both ETA and Voltas have relationships with factories in India, giving them an edge on prices. A Project that has accepted an Indian subcontractor is amenable to Indian quality products that are currently unbeatable in prices.
\item International European Firms, such as Thermo 
\item Local Companies, backed by serious money  for example Al-Futeim Group's Carillion. These Companies work both on price as well as their ability to network.
\end{itemize}

\subsection*{Clients}

Clients are mostly Main Contractors, although direct nomination by Owners and Project Manager's is a possibility. Other stake holders in the approval process are Consultant's and Owner representative firms. Very little work so far has been carried out directly with Government Organizations.

\subsection*{Image}

Although one can say the image we want to project is one of professionalism, that is too general. In order to lay out the requirements and needs of the amrket and translate them to a Product one need to rethink the total operations.

\textbf{During project Execution.}\quad
The most important image we need to project is one of \textit{effortless} execution. What I mean with effortless execution, is that we need to execute all the stages of the Project with proactive resolution of issues, meeting of \textit{all} deadlines, from a promise to deliver a document or letter to installation works, to fitting our operations with those of the Main Contractor.

In order to achieve this a number of prerequistes need to be met, which we will analyze a bit later, a simple interpretation of this requirement is the need to stay ahead rather than be behind the Design team or teh Main Contractor's Team. Simply translated add well organized resources at the early stages of the Project.

\texbf{Paper work.}\quad During the life time of the Project, until perhaps the 60\% stage performance is mostly judged on \textit{paper work}. Method Statement for a new Project can be delivered in a well \textit{typeset manner} within the first month of the Project. they are repeated from Project to Project and the reason why the are re-written every time, is that no-body has perfected them to the point that they satisfy a different set of Project Managers. If the sections and the write up is ready, it is quite possible to achieve this and \textit{automate} it. On a large Project, what should be a minor cost translates to a constant hassle that probably is equal to the cost of one Engineer for the duration of the Project. With the right investment in preparing these upfront it is possible to reduce this cost substantially and improve our image. 

\textit{Submittals}\quad
This is normally a horror story. It is not unknown that it can take up to a year to get some materials approved. The problem here starts from getting quotations, getting the supplier approved from everyone involved and then writing up the submittal (normally the supplier will draft it up---which in many instances due to format problems, it gets rejected). 

\textit{Drawings}\quad Here one needs to lift the barriers of quality. Good drawings save money and accelerate the works. The duration for drawing production should be worked out on 3 months for first iteration and a six month window for A status full set of drawings. If there are change orders, resource can be added accordingly at a later stage, plus claiming overheads for the Engineering Team.

``Cats are intended to teach us that not everything in nature has
a function.''

Garrison Keillor (1942–)

\section{Improving Production and Focusing on our Product}

Consider the possibilities. A recent survey of published results by manufacturing
and service companies that have applied constraint management
methods effectively shows:\footnote{Source: Mabin, Victoria J. and Steven J. Balderstone,
The World of the Theory of Constraints:A Review of the International Literature, St. Lucie Press, Boca Raton, FL, 2000.}

\begin{enumerate}
\item A mean reduction in lead times of 70\%

\item A mean reduction in manufacturing cycle times of 65\%

\item A mean improvement in due-date performance of 44\%

\item Mean inventory reductions of 49 percent

\item A mean combined financial improvement (revenue, throughput,
profit) of 76.\%

\end{enumerate}

I am convinced that as applied to the Construction Industry this might even be an understatement. I am also convinced that the actual methodology will have to be improved, but one wonders why this is not happening.

\begin{figure}
\leftskip20pt \rightskip20pt
Peninsula Daily News
(Associated Press), Port Angeles, WA,

Monday, Aug. 2, 1999, p. A5

Delays in spray paint delivery curtail Forest Service harvests

Associated Press

Logging in federal forests is down by as much as 25 percent this year in regions
outside the Northwest because the U.S. Forest Service can’t find enough of the paint
it needs to mark trees for cutting. The agency says that it will try to make up the
logging deficit by next year at the latest.
The paint shortage so far has not had an impact on timber sales in Washington
and Oregon. “We still have paint, but we’re running low — and we have an emergency
order in,” spokesman Rex Holloway said.
Timber sales in the five-state Rocky Mountain region are down between 15 and
25 percent because of the paint shortage, forest officials said. Many of the other nine
regions are in “roughly the same ballpark,” although they could catch up before the
end of the year, said Ann Bartuska, director of forest management for the Forest
Service in Washington, D.C. “People are coping perfectly well,” she added.
Some foresters are marking trees the old fashioned way while they wait for the
paint — using a hatchet to notch the trees, Bartuska said.
The agency on May 15 stopped using, oil-based paint to mark the trees it plans
to cut down, after workers blamed the paint on an increase in miscarriages and other
ailments. But Bartuska said the transition to water-based paint was more difficult
than expected. There was an explosion at a paint factory, procurement delays by the
General Services Administration and delays in getting the forest workers’ union to
approve the new paint.
“Everything that could go wrong has gone wrong,” she said, adding that the Forest
Service is rushing as much paint out to forests as it can.
Timber sales in the Rocky Mountain region — South Dakota, Nebraska, Kansas,
Wyoming and Colorado — will be down between 30 million and 51 million board
feet this year, regional forester Lyle Laverty said, in a letter earlier this month. That
disclosure prompted a letter last week from six senators to Forest Service Chief Make
Dombeck.
``These downfalls are not acceptable,''  the senators from Wyoming, South Dakota
and Colorado wrote.

\copyright 1999, The Associated Press. All rights reserved. Reprinted with permission.

\end{figure}


To know with confidence that a local decision truly advances the goal of
the company (which we assume for most organizations is to
make money), we need to understand how such decisions actually impact
the global financial performance of the organization---if they do at all. What
management needs is a new set of yardsticks that enhances decision making
at all levels. These yardsticks should provide a clear, unequivocal connection
between local decisions and global performance. They should motivate functional
managers to make the right exploitation and subordination decisions
--- the decisions that improve the whole organization’s performance, not just
their own department’s. And these yardsticks must be simple and easily
understood by everyone.






