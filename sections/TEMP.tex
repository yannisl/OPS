\begin{comment}
% !TEX TS-program = pdflatex
% !TEX encoding = UTF-8 Unicode

\documentclass[a4paper, oneside, justified=true, sfsidenotes, titlepage=true]{tufte-book} % for a long document


%% We use the inputenc package and babel
%% to handle the correct encoding

\usepackage[utf8]{inputenc} % set input encoding to utf8
\usepackage[english]{babel}[2005/11/23]
\usepackage{lipsum}
\usepackage{listings}
\usepackage{lineno}
\usepackage{xcolor}
\usepackage{colortbl}
\usepackage{booktabs}

%% Bibliography
%% We use natbib
\usepackage{natbib}
\bibliographystyle{plainnat} 
\bibpunct{(}{)}{;}{a}{,}{,}


\usepackage{amsmath}[2000/07/18] %% Displayed equations
\usepackage{amssymb}[2002/01/22] %% and additional symbols

\usepackage{alltt}[1997/06/16]   %% boilerplate, credits, license
\usepackage{epigraph}
% use courier acrobat fonts rather than the type writer font
%\usepackage{courier}

\usepackage{lettrine}

% For graphics / images
\usepackage{graphicx}
\setkeys{Gin}{width=\linewidth,totalheight=\textheight,keepaspectratio}
\graphicspath{{graphics/}}


\usepackage{pgfplots}
\pgfplotsset{compat=1.3}

%% enable draft to be inserted
%% see later commands

%\usepackage[some]{background}


%overwrite tufte and have numbering





%%hyperref
\hypersetup{urlcolor=blue, colorlinks=true}  % Colours hyperlinks in blue, but this can be distracting if there are many links
\urlstyle{sf}  %set url as sans-serif font (in url? find how to do with hyperref)

%% for short experiments
\usepackage{pifont}
 \newcommand\lorem{Fusce adipiscing justo nec ante. Nullam in enim.
 Pellentesque felis orci, sagittis ac, malesuada et, facilisis in,
 ligula. Nunc non magna sit amet mi aliquam dictum. In mi. Curabitur
 sollicitudin justo sed quam et quadd. \par}



% Typesets the font size, leading, and measure in the form of 10/12x26 pc.
\newcommand{\measure}[3]{#1/#2$\times$\unit[#3]{pc}}

% Macros for typesetting the documentation

\newcommand{\fox}{\medskip  "The quick brown fox jumps over the lazy dog"\medskip} % mini lorem
\newcommand{\dogs}{The sentence is often mistakenly rendered as "The quick brown fox jumped over the lazy dog," which does not include an s. However, this can be corrected by typing: "The quick brown fox jumped over the lazy dogs".}


\newcommand{\hlred}[1]{\textcolor{Maroon}{#1}}% prints in red

\newcommand{\hlmaroon}[1]{\textcolor{Maroon}{#1}}%print maroon

\newcommand{\hangleft}[1]{\makebox[0pt][r]{#1}}


\newcommand{\sourceatright}[2]{{%
  \unskip\nobreak\hfil\penalty100
  \hskip#1\hbox{}\nobreak\hfil{#2}
  \parfillskip 15pt \par}}


\newcommand{\hairsp}{\hspace{1pt}}% hair space
\newcommand{\hquad}{\hskip0.5em\relax}% half quad space
\newcommand{\TODO}{\textcolor{red}{\bf TODO!}\xspace}

\newcommand{\ie}{\textit{i.\hairsp{}e.}\xspace}
\newcommand{\eg}{\textit{e.\hairsp{}g.}\xspace}


%\newcommand{\na}{\quad--}% used in tables for N/A cells

\providecommand{\XeLaTeX}{X\lower.5ex\hbox{\kern-0.15em\reflectbox{E}}\kern-0.1em\LaTeX}
\newcommand{\tXeLaTeX}{\XeLaTeX\index{XeLaTeX@\protect\XeLaTeX}}
% \index{\texttt{\textbackslash xyz}@\hangleft{\texttt{\textbackslash}}\texttt{xyz}}

%% define backsalsh
%% you can also use \textbackslash
\newcommand{\tuftebs}{\symbol{'134}}% a backslash in tt type in OT1/T1

\newcommand{\doccmdnoindex}[2][]{\texttt{\tuftebs#2}}% command name -- adds backslash automatically (and doesn't add cmd to the index)

\newcommand{\doccmddef}[2][]{%
  \hlred{\texttt{\tuftebs#2}}\label{cmd:#2}%
  \ifthenelse{\isempty{#1}}%
    {% add the command to the index
      \index{#2 command@\protect\hangleft{\texttt{\tuftebs}}\texttt{#2}}% command name
    }%
    {% add the command and package to the index
      \index{#2 command@\protect\hangleft{\texttt{\tuftebs}}\texttt{#2} (\texttt{#1} package)}% command name
      \index{#1 package@\texttt{#1} package}\index{packages!#1@\texttt{#1}}% package name
    }%
}% command name -- adds backslash automatically

%% doc commands
% % adds hem automatically to index

\newcommand{\doccmd}[2][]{%
  \texttt{\hlred{\tuftebs#2}}%
    \ifthenelse{\isempty{#1}}%
    {% add the command to the index
      \index{#2 command@\protect\hangleft{\texttt{\tuftebs}}\texttt{#2}}% command name
      % \marginpar{\hlred{#1}}
    }%
    {% add the command and package to the index
      \index{#2 command@\protect\hangleft{\texttt{\tuftebs}}\texttt{#2} (\texttt{#1} package)}% command name
     \index{#1 package@\texttt{#1} package}\index{packages!#1@\texttt{#1}}% package name
    }%
}% command name -- adds backslash automatically



%% doc options
%
\newcommand{\docopt}[1]{\ensuremath{\langle}\textrm{\textit{#1}}\ensuremath{\rangle}}% optional command argument


\newcommand{\docarg}[1]{\textrm{\textit{#1}}}% (required) command argument
\newenvironment{docspec}{\begin{quotation}\ttfamily\parskip0pt\parindent0pt\ignorespaces}{\end{quotation}}% command specification environment


\newcommand{\docenv}[1]{\texttt{#1}\index{#1 environment@\texttt{#1} environment}\index{environments!#1@\texttt{#1}}}% environment name


\newcommand{\docenvdef}[1]{\hlred{\texttt{#1}}\label{env:#1}\index{#1 environment@\texttt{#1} environment}\index{environments!#1@\texttt{#1}}}% environment name

%% Provide a  command to add packages to index
%
\newcommand{\docpkg}[1]{\hlred{\texttt{#1}}\index{#1 package@\texttt{#1} package}\index{packages!#1@\texttt{#1}}
\marginpar{\hlred{#1~package}}
}% package name

%% Provide a command for document cls - do not add to index
%
\newcommand{\doccls}[1]{\texttt{#1}}% document class name


\newcommand{\docclsopt}[1]{\texttt{#1}\index{#1 class option@\texttt{#1} class option}\index{class options!#1@\texttt{#1}}}% document class option name
\newcommand{\docclsoptdef}[1]{\hlred{\texttt{#1}}\label{clsopt:#1}\index{#1 class option@\texttt{#1} class option}\index{class options!#1@\texttt{#1}}}% document class option name defined
\newcommand{\docmsg}[2]{\bigskip\begin{fullwidth}\noindent\ttfamily#1\end{fullwidth}\medskip\par\noindent#2}
\newcommand{\docfilehook}[2]{\texttt{#1}\index{file hooks!#2}\index{#1@\texttt{#1}}}
\newcommand{\doccounter}[1]{\texttt{#1}\index{#1 counter@\texttt{#1} counter}}

%% margin document commands
%
\newcommand{\margindoc}[1]{\marginpar[]{\doccmd{#1}}}

\newcommand{\sidebarcolor}[1]{\textcolor{darkgray}{\sidenote[]{#1}}}

%% alias of \doccmnd
\newcommand{\cmd}[1]{\doccmd{#1} \marginpar[]{\hlred{\doccmd{#1}}}}


%% Set up the epigraph to be a bit wider
\setlength{\epigraphwidth}{8cm} 
\setlength{\epigraphrule}{0pt}
\newcommand{\theepigraph}[2]{\epigraphhead[30]{\epigraph{#1}{\textit{#2}}}}







\newcommand{\latex}{\LaTeX}
\newcommand{\tex}{\TeX}
\newcommand{\texbook}{\tex book\  }

\newcommand{\bs}{$backslash$}

\newcommand{\BC}{\textsc{bc}}
\newcommand{\AD}{\textsc{ad}}

\title{MEP CONSTRUCTION STRATEGY}
\author[Y Lazarides]{HEE-Specon}
\date{July 2010} % Delete this line to display the current date


%\newcommand\bhrule{\typeout{------------------------------------------------------------------------------}}


%% Change some of the looks
%%% CHAPTER STYLES AND SECTION STYLES
\pagestyle{fancy}
\renewcommand{\chaptertitlename}{CHAPTER}
\renewcommand{\chaptermark}[1]%
{\markboth{\MakeUppercase{\chaptertitlename\ CHAPTER \thechapter\ #1}}{}} %no dot here
\renewcommand{\sectionmark}[1]%
{\markright{\MakeUppercase{\thesection~\ #1}}}
\renewcommand{\headrulewidth}{0.5pt}
\renewcommand{\footrulewidth}{0pt}
\newcommand{\helv}{%
\fontfamily{phv}\fontseries{b}\fontsize{9}{11}\selectfont}
\fancyhf{}
\fancyhead[LE,RO]{TESTING \thepage}
\fancyhead[LO]{\helv \rightmark}
\fancyhead[RE]{\helv TEST \leftmark}


%% We generate the index
%% on a fresh installation run makeindex once as well

\usepackage{makeidx}
\makeindex


%\usepackage{fancyvrb}
%\usepackage{float}
%\usepackage{wrapfig}

%% Using amsmath and amssymbols for 
%% mathematics
%% they have better macros that those provided in LaTex or TeX
%
\usepackage{amsfonts}

\usepackage{amsmath}
\usepackage{amssymb}[2002/01/22] %% and additional symbols


%% Eventually beginning the document
%% 


%% Set some local commands and colors
\definecolor{green}{rgb}{0.1,0.1,0.1}
\newcommand{\done}{\cellcolor[gray]{0.9}done}  %{0.9}

\usepackage{titlesec}
\usepackage{titletoc}
\renewcommand{\chaptertitlename}{The Chapter}
\titleformat{\chapter}%
  [display]% shape
  {\relax\ifthenelse{\NOT\boolean{@tufte@symmetric}}
              {\begin{fullwidth}}{}}% format applied to label+text
  {\Large \textsf{\chaptertitlename} \Large \textsf{\thechapter}}% label sans-serif number 
  {0pt}% horizontal separation between label and title body
  {\huge\textsf}% before the title body \itshape in tufte
  [\ifthenelse{\NOT\boolean{@tufte@symmetric}}{\end{fullwidth}}{}]% after the title body


%% experiment
\lhead{\nouppercase{\rightmark} (\nouppercase{\leftmark})}
  \chead{}
  \rhead{}
  \lfoot{\today}
  \cfoot{}
  \rfoot{\thepage}
  \renewcommand{\headrulewidth}{0.4pt}
  \renewcommand{\footrulewidth}{0.4pt}

  \renewcommand{\chaptermark}[1]{%
  \markboth{#1}{}}

\end{comment}