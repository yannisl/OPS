\newcounter{inc}{0}
\def\inc{\addtocounter{inc}{1}\theinc}


\chapter{Materials Control Department}


The Materials Control Department on a Project is the responsible department that ensures that materials are purchased at the \textit{least possible price, they
meet the Project quality standards  and that they are delivered on time}. 

\section*{Materials Planning}
This is the most difficult phase of teh procurement cycle and if not planned
properly the MCD department will be in \textit{panic mode} until the end of the
Project. At the beginning of the Project a Materials Control Sheet is created
listing all the material requirements of the Project by categories. Categories
are normally split in such a way that a category is materials purchased from a single
vendor.

At the beginning of the Project at the discretion of the Area Manager and the Project
Director, the Engineering Office as well as all the Engineers and perhaps the
CAD Office will contribute to material take-offs. The Engineering Office
will also handle enquiries and discussions with Suppliers for items where
a strong technical input is required. Here is a short list, however what is important
is to generate such lists comprehensively at the beginning of the Project. It is the 
responsibility of the MCD Manager to monitor progress and to produce weekly reports.

\begin{fullwidth}
\begin{table}[htbp]
\vspace{0.5cm}
\begin{tabular}{clllp{3cm}}
\toprule
Item  &Responsibility &Technical submittal &Remarks\\
\midrule
\inc &AHUs  & Engineering Office & Engineering Office & take-offs by Section Engineers\\
\inc &fcus  & Engineering Office & Engineering Office & take-offs by Section Engineers\\
\inc &Chillers  & Engineering Office & Engineering Office & take-offs by Section Engineers\\
\inc &Pumps & Engineering Office & Engineering Office & take-offs by Section Engineers\\
\inc &Package Units  & Engineering Office & Engineering Office & take-offs by Section Engineers\\
\inc &Fans  & Engineering Office & Engineering Office & take-offs by Section Engineers\\
\inc &ECUs  & Engineering Office & Engineering Office & take-offs by Section Engineers\\
\bottomrule
\end{tabular}
\caption{Mechanical long lead items}
\label{longleaditems}
\end{table}
\end{fullwidth}

It is noteworthy, that all the equipment listed above need to have static calculations in order for orders to be finalized. This is not always possible to be carried out at the beginning of the Project
and is best to agree with the Supplier cut-off dates for the supply of this information. In general
if you adhere to the following process things are easier\sidenote{With software these calculations are not difficult to produce, if the Tender drawings are reasonably well co-ordinated.}:

\begin{enumerate}
\item Send out enquiries early based on Tender documents.
\item Narrow down the price with one Supplier.
\item Work with the Supplier to obtain selections and submittals.
\end{enumerate}

Although it is important not to lose time and long-lead items need to be ordered as early as possible
it is also very important to act quickly and get approvals and orders out for first fix materials. 

\begin{fullwidth}
\begin{table}[htbp]
\vspace{0.5cm}
\begin{tabular}{clllp{3cm}}
\toprule
Item  &Responsibility &Technical submittal &Remarks\\
\midrule
\inc &anchors  & Engineering Office & Engineering Office & take-offs by Section Engineers\\
\inc &threaded rods  & Engineering Office & Engineering Office & take-offs by Section Engineers\\
\inc &supports  & Engineering Office & Engineering Office & take-offs by Section Engineers\\
\inc &unistrut  & Engineering Office & Engineering Office & take-offs by Section Engineers\\
\inc &insulation inserts  & Engineering Office & Engineering Office & take-offs by Section Engineers\\
\inc &conduit systems  & Engineering Office & Engineering Office & take-offs by Section Engineers\\
\bottomrule
\end{tabular}
\caption{Mechanical first fix items}
\label{firstfixitems}
\end{table}
\end{fullwidth}

Table \ref{firstfixitems} if handled properly and quickly they enable operations to start on Site.
There is no benefit in focusing first on pipes and fittings, if the above will not have an approval
and there is no stock on site to start the works. These materials should be closed in the first 3-4 weeks of the Project. 


\begin{fullwidth}
\begin{table}[htbp]
\vspace{0.8cm}
\begin{tabular}{clllp{3cm}}
\toprule
Item  &Responsibility &Technical submittal &Remarks\\
\midrule
\inc &drainage pipes  & Engineering Office & Engineering Office & take-offs by Section Engineers\\
\inc &chilled water pipes  & Engineering Office & Engineering Office & take-offs by Section Engineers\\
\inc &fire protection pipes  & Engineering Office & Engineering Office & take-offs by Section Engineers\\
\inc &H\&C water piping & MCD & MCD/EO & BOQ Section Engineers\\
\inc &Cable trays & MCD & MCD/EO & BOQ Section Engineers\\
\inc &Cable ladders & MCD & MCD/EO & BOQ Section Engineers\\
\inc &ductwork & MCD & MCD/EO & BOQ Section Engineers\\
\bottomrule
\end{tabular}
\caption{Mechanical first fix items (second tier)}
\label{firstfixitems}
\end{table}
\end{fullwidth}


\section*{Second and Third Fix Materials}
Depending on the Project and its requirements second and third fix materials are tackled next. 
The list below is not comprehensive, but should mre or less be prioritized as shown. Keep in
mind what you need to complete the installation and what is affecting follow-up trades such as
the Main Contractor cosing walls or ceilings.

\begin{fullwidth}
\begin{table}[htbp]
\vspace{0.8cm}
\begin{tabular}{clllp{3cm}}
\toprule
Item  &Responsibility &Technical submittal &Remarks\\
\midrule
\inc &drainage specialties  & Engineering Office & Engineering Office & take-offs by Section Engineers\\
\inc &interceptors& Engineering Office & Engineering Office & take-offs by Section Engineers\\
\inc &manhole covers& Engineering Office & Engineering Office & take-offs by Section Engineers\\
\inc &paddle-flanges& Engineering Office & Engineering Office & take-offs by Section Engineers\\
\inc &special sleeves& Engineering Office & Engineering Office & take-offs by Section Engineers\\
\bottomrule
\end{tabular}
\caption{Drainage materials}
\label{firstfixitems}
\end{table}
\end{fullwidth}

As for the drainage materials, focus is being maintained as to what is needed next. For HVAC is making sure that what is required in terms of ductwork and piping.

\begin{fullwidth}
\begin{table}[htbp]
\vspace{0.8cm}
\begin{tabular}{clllp{3cm}}
\toprule
Item  &Responsibility &Technical submittal &Remarks\\
\midrule
~ &\textbf{Ducted systems}&  &  & \\
\inc &Fire Dampers& Engineering Office & Engineering Office & take-offs by Section Engineers\\
\inc &Volume Dampers & Engineering Office & Engineering Office & take-offs by Section Engineers\\
\inc &Motorized Dampers & Engineering Office & Engineering Office & take-offs by Section Engineers\\
\inc &Access Doors & Engineering Office & Engineering Office & take-offs by Section Engineers\\
\inc &Sound attenuators & Engineering Office & Engineering Office & take-offs by Section Engineers\\
\inc &Flexible ducts & Engineering Office & Engineering Office & take-offs by Section Engineers\\
\inc &Flexible connectors & Engineering Office & Engineering Office & take-offs by Section Engineers\\
\inc &Grilles \& Diffusers & Engineering Office & Engineering Office & take-offs by Section Engineers\\
\inc &Louvres  & Engineering Office & Engineering Office & take-offs by Section Engineers\\
\inc &Sand-trap louvres  & Engineering Office & Engineering Office & take-offs by Section Engineers\\
\midrule
~ &\textbf{Piping Systems} &  &  &  \\
\inc &Valves  &EO   &EO  &EO/CAD \\
\inc &Flanges &  &  &\\
\inc &Gaskets &  &  &\\
\inc &Bolts   &  &  &\\
\inc &Insulation & & &\\
\inc &Insulation accessories & & &\\
\inc &Chemical treatment & & &\\
\inc &Pressurization units & & &\\
\inc &Expansion vessels & & &\\
\inc &De-aerators & & &\\
\midrule
~ &\textbf{BMS \& Controls} &  &  &  \\
\inc &BMS General & & &\\
\inc &BMS Graphics & & &\\
\inc &BMS Cables   & & &\\
\bottomrule
\end{tabular}
\caption{HVAC material lists}
\label{firstfixitems}
\end{table}
\end{fullwidth}

If everything has been co-ordinated properly, orders are placed on agreed staged deliveries or to
draw as per Site requirements. When you place orders full at the beginning of the Project you
save considerable trouble, you manage less documents, less submittals etc. In general on most
jobs, there will be about 150-160 categories of materials. On a target to close between 2-3 categories daily in the first 90 days of the project you will need 3-4 people to process, plus of course the full site Team should be contributing to this. It is the responsibility of the Project Manager to arrange a meeting where \textit{names} are put next to each material to ensure action and responsibilities are clear cut and the materials are processed as fast as possible.


\section*{Stock control for consumables}

Two bin systems are common on assembly and moving manufacturing lines where components are added to the product or item being built. The two bin system is just like its name suggests, it is composed of two bins which are full of components or materials to start. As production commences one bin is drawn down of materials and the other bin, which is still full, acts as the buffer or safety stock.

When the first bin is completely depleted the worker or assembly line worker switches to the other bin, similar to a FIFO system. The switch of bins can be interpreted as a kanban signal for the supply process of that particular component to manufacture or supply the component just in time before the second bin runs out of material. The kanban signal can also be generated half way throughout the first bin, depending on lead times for the component to be supplied. 

This system in a way is similar to the EOQ inventory model with safety stock. It is a very common system used in vehicle manufacturing plants. The size or number of components in each bin is usually determined using the EOQ inventory model or a time period model. 