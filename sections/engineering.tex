\chapter{Engineering}

In our line of work \textit{design} is both ill-defined as well as generally poorly
misunderstood and applied. On most Projects the Owner would have appointed a
Consultant who will produce a design and give it out to tender. In many cases
this is incomplete and supplemented with every conceivable type of clause in a
specification to limit claims by the Contractor. On other type of Contracts we
might have design responsibility for the full works. These are called design-build contracts. In either case a substantial amount of engineering needs to be developed
to complete the design, procure the materials, install the works and putting the
plant into operation.



\section*{The Engineering Department}

On Projects that are not design-build an Engineering Department is set-up with
an Engineering Manager. He is normally assisted with Design Engineers. In addition
he is the ultimate responsible person for overseeing the works of the CAD Office.

\section{Design Audit}

At the start of the Project a quick design audit is undertaken to review the
current design by the Engineer and to identify areas of concerns, missing information and opportunities for savings in costs. This should be summarized in a report with
full details. The same document can then progressively be modified to record all
changes and calculations as works progress.

\subsection*{Selection of equipment}

Equipment selections and submittals fall within the work that the Engineering Department undertakes. As these works might take time to implement the Project Director at the early stages of a Project allocate responsibilities to Senior
Engineers to assist. Some equipment might need calculations before they are ordered as for example fans, pumps, cables, electrical boards and similar items.


\section*{Drawings}

The production of drawings should be done in the \textit{fastest} way possible. 
This is important for two reasons:

\begin{enumerate}
\item As at the beginning of the Project, information might be missing, this is
the best time to record delays, before any \textit{concurrent delays} occur. In many instances the Project Managers or Engineer might issue an information schedule, i.e, a time table listing all the missing information and when this will be issued. Unless this is covered with the Clause (14) programme, one should only accept with the reservation that the Contractot has a right to claim for late information. The Commercial Manager for certain would need to respond. Although
late information for sanitary fixings might not delay the Project it may delay the
drawings and the opening required to accommodate floor mounted units and similar items. If the Engineer has issued a set of IFC drawings no further delays without claims would be acceptable.
\item If works start late on site due to late issue of Shop Drawings or unavailabilty of materials due to late issue of information we would need to add
tradesment to accelerate the works and keep up with the programme. It is always easier to add one CAD Operator than to add 100 tradesmen and more cost effective, given the inefficiencies of disruption and crowded work faces.

\end{enumerate}

\section*{How big a team}

Although, this will depend on the quality of the original design, the number of services involved, the nature of the Project and if the Building has areas where
the design is repetitive a good formula to use is the following:

\begin{equation} n = \sum_{n=1}^{n=k}\frac{f_1}{100}\times C + \frac{f_2}{100}\times C +\cdots+\frac{f_k}{100}\times C_k
\end{equation}

The mimimum size Team should never be less than 16 on Projects that are greater
than QR~200~million.



\section*{Co-ordination}


Possibly more than 70\% of all issues that cause work to stop or to move 
inefficiently can be attributed to poor co-ordination. Although the original
designers bear responsibility for primary co-ordination, we bear responsibility
for co-ordinating the services and for raising the alarm when there are problems
with primary co-ordination.
\begin{marginfigure}
\tikz
\node [forbidden sign,line width=1ex,draw=red,fill=white, scale=2.0] {CLASHES};
\end{marginfigure}

\subsection*{Co-ordination sequence}
At the start of a Project co-ordination needs to follow a sequence of works and
although one will need to go through a number of iterations before the drawings
can be finalized the following sequence will work in most typical cases:

\begin{enumerate}
\item If the ceiling voids have large beams, consider locating the fire mains within the
void provided by the beams to conserve space. Fire mains can easily be sleeved
through and is a good solution, especially in car parks.
\item Drainage pipes should be located hard against the nearest beam and sloped as required, consider changing direction if necessary to have drainage pipes run
shorter routes.
\item It is normal where there is drainage pipes to have cold and hot water. Following a similar route for these pipes would ease co-ordination. 

\item Cable trays should be similarly zoned in parallel with walls. Preferably
no other services should run under or over cable trays to enable clearances for 
pulling cables.

\item Ductwork is a special case and if possible should be run at the lowest level
with flexibles \textit{on the side of the ducts}.

\end{enumerate}

\subsection*{Co-ordination in shafts}

\begin{enumerate}
\item Shaft co-ordination should start early and if possible the first type of co-ordination to take place. This is important as any crossover of services
would seriously reduce ceiling void heights.
\item Always consider providing a platform in shafts. Although this might marginally add to costs, it can speed-up construction tremendously and enable easy hoisting of piping through the shafts.
\item Method of accessible shafts, need to be agreed with stakeholders.
\item For typical plumbing shafts on high rise buildings and or hotels and
apartments consider making an early mock-up--not necessarily in situ--to 
optimize the installation and work out small details such as expansion.
\item Allow adequate space behind ducts and pipes for access to install insulation. This is a normally an overlooked aspect.
\end{enumerate}


\subsection*{Co-ordinators}

Each CAD operator has responsibilty to check his work in terms of
completeness and co-ordination.

\subsection*{Common pitfalls}

\begin{enumerate}
\item One service obstructing another, such as a BMS control box installed too near a fan-coil unit, so that it cannot be opened. This is also common with MCC
boards being too big and cannot be accomodated in Plantrooms.

\item Condensate drains on mechanical equipment that have been - missed on drainage drawings. This can be avoided by showing condensate drains and fan-coil
units on a separate layer and importing this layer on both the VAC drawings
as well as the drainage drawings. Similarly for any equipment requiring power supplies etc.

\item Not allowing an adequate clearance both under or over sectional water tanks.
		This is also a common error by Consultants.

\item Very low plinths for AHUs with high static pressures. This can also happen
		when plinths are designed from structural drawings and then waterproofing is about
		50 mm high ending up with plinths that are only 50 mm high. 

\item Not allowing properly for the curvature of cables when entering switchboards
		or for panels to be ordered bottom entry but the cables are top entry. This is a 
		result of poor scheduling and detailing.

\end{enumerate}



\section*{Professional Drawings}

One can easily recognize a professional drawing when he sees one, but it is 
very difficult to explain what makes a drawing professional.
These are some parameters:

\begin{enumerate}
\item All information required to execute the works is shown on the drawings.
\item Details are specific for the Project and are meaningful.
\item Schedules.
\item Cross referencing.
\item Good choice of pens and styles.
\item The drawing has \textit{dimensions}. 
\item The drawings include sections and larger scale details.
\item The drawing is printed at the right scale.
\end{enumerate}














