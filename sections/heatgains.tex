\ctable[]{rlcc}{
\tnote{for the abstraction reaction,$\fam0 Mu+HX \rightarrow MuH+X$.}
\tnote[b]{1 degree${} = \pi/180$ radians.}
\tnote[c]{this is a particularly long note, showing that
footnotes are set in raggedright mode as we don't like
hyphenation in table footnotes.}
}{ \FL
& & $\fam0 H(Mu)+F_2$ & $\fam0 H(Mu)+Cl_2$ \ML
&$\beta$(H) & $80.9^\circ$\tmark[b] & $83.2^\circ$ \NN
&$\beta$(Mu) & $86.7^\circ$ & $87.7^\circ$ \LL
}


\ctable[]{c>{\raggedright}Xc>{\raggedright}X}{
\tnote{footnotes are placed under the table}
}{ \FL
\multicolumn{4}{c}{Example using tabularx} \ML
\multicolumn{2}{c}{Multicolumn entry!} & THREE & FOUR \NN
\cmidrule(r){1-2}\cmidrule(rl){3-3}\cmidrule(l){4-4}
one&
The width of this column depends on the width of the
table.\tmark &
three&
Column four will act in the same way as
column two, with the same width. \LL
}




\lstloadlanguages{Pascal, Ada}
\lstset{language=Pascal,commentstyle=\scriptsize}
A "for" loop in C:
\begin{lstlisting}[keywordstyle=\underbar]
int sum;
int i; /*for loop variable*/
sum=0;
for (i=0;i<n;i++) {
  sum += a[i];
}
\end{lstlisting}
Now the same loop in Ada:
\begin{lstlisting}[language=Ada]
Sum: Integer;
-- no decl for I necessary
Sum := 0;
for I in 1..N loop
Sum := Sum + A(I);
end loop;
\end{lstlisting}


\appendix{Common Requirements}
\section{Highlighting text}



\newenvironment{Description}
{\begin{list}{}{\let\makelabel\Descriptionlabel
\setlength\labelwidth{40pt}%
\setlength\leftmargin{\labelwidth+\labelsep}}}%
{\end{list}}
\newcommand*\Descriptionlabel[1]{\textsf{#1:}\hfil}
\begin{Description}
\item[Description]
Returns from a function. If issued at top level,
the interpreter simply terminates, just as if
end of input had been reached.
\item[Errors] None.
\item[Return values]
\mbox{}\\
Any arguments in effect are passed back to the
caller.
\end{Description}


\chapter{Multicolumn Layouts}
\begin{multicols}{2}
Here is some text to be distributed over
several columns. 
narrow try typesetting ragged right.
\lipsum
\end{multicols}

\chapter{Climate}
The variations in temperature, humidity and wind occurring throughout the world are due
to several factors the integration of which, for a particular locality, provides the climate
experienced. There is, first, a seasonal change in climatic conditions, varying with latitude
and resulting from the fact that, because the earth's axis of rotation is tilted at about 23.5 ~
to its axis of revolution about the sun, the amount of solar energy received at a particular
place on the earth's surface alters throughout the year. The geography of the locality
provides a second factor, influential in altering climate within the confines imposed by the
seasonal variation.

Figure 5.1 illustrates the geometrical considerations which show that, at a particular
latitude, the earth receives less solar radiation in winter than it does in summer. The
importance of this in its effect on the seasons stems from the fact that, for all practical
purposes, the sun is the sole supplier of energy to the earth.

The geography of a place determines how much solar energy is absorbed by the earth,
how much is stored and how readily it is released to the atmosphere. The atmosphere is
comparatively transparent to the flux of radiant solar energy (termed \emph{insolation\/}) but the
land masses which receive the energy are opaque to it and are fairly good absorbers of it,
although this depends on the reflectivity of the surface. This means that the thermal energy
from the sun warms up land surfaces on which it falls. Some of this energy travels inwards
and is stored in the upper layers of the earth's crust; some is convected to the atmosphere
and some is re-radiated back to space, but at a longer wavelength (about 10 micrometres),
since its mean surface temperature is very much less than that of the sun. Four-fifths of the
earth's surface is water, not land, and water behaves in a different fashion as a receiver of
insolation, being partially transparent to it; consequently, the energy is absorbed by the
water in depth, with the result that its surface temperature does not reach such a high value
during the daytime. On the other hand, at night, heat is lost from the land to the sky much
more rapidly since less was stored in its shallow upper crust than was absorbed and stored
in the deeper layers of water. The result is that land-surface temperatures tend to be lower
at night than are water-surface temperatures. It is evident from this that places in the
middle of large land masses will tend to have a more extreme annual variation of temperature
than will islands in a large sea. Thus, the climate of places on the same latitude can vary
enormously. To realise this we have but to compare the temperate seasons experienced in
the British Isles with the extremes suffered in Central Asia and Northern Canada at about
the same latitude. The exchanges of radiant energy cited above as responsible for the
differences in maritime and continental climates are complicated somewhat by the amount
of cloud. Cloud cover acts as an insulating barrier between the earth and its environment;
not only does it reflect back to outer space a good deal of the solar energy incident upon
it but it also stops the passage of the low-frequency infra-red radiation which the earth
emits. In addition, the quantity of carbon dioxide in the atmosphere reduces the emission
of infra-red radiation. Mountain ranges also play a part in altering the simple picture,
presented above, of a radiation balance.


The effect of the unequal heating of land and sea is to produce air movement. This air
movement results in adiabatic expansions and compressions taking place in the atmosphere,
with consequent decreases and increases respectively in air temperature. These temperature
changes, in turn, may result in cloud formation as values below the dew point are reached.
One overall aspect of the thermal radiation balance is prominent in affecting our weather
and in producing permanent features of air movement such as the Trade Winds and the
Doldrums. This is the fact that, for higher latitudes, the earth loses more heat to space by
radiation than it receives from the sun but, for lower latitudes, the reverse is the case. The
result is that the lower latitudes heat up and the higher ones cool down. This produces a
thermal up-current from the equatorial regions and a corresponding down-current in the
higher latitudes. While this is true for an ideal atmosphere, the fact that the earth rotates
and that other complicating factors are present means that the true behaviour is quite
involved and not yet fully understood.


\section{ Winds}
It might be thought that wind flowed from a region of high pressure to one of low pressure,
following the most direct path. This is not so. There are, in essence, three things which
combine to produce the general pattern of wind flow over the globe. This pattern is
complicated further by local effects such as the proximity of land and sea, the presence of
mountains and so on. However, the three overall influences are:


(i) the unequal heating of land and sea;

(2)the deviation of the wind due to forces arising from the rotation of the earth about its
axis;
(iii) the conservation of angular momentum~a factor occurring because the linear velocity
of air at low latitudes is less than at high latitudes.
The general picture of wind distribution is as follows. Over equatorial regions the weather
is uniform; the torrid zone is an area of very light and variable winds with frequent calms,
cloudy skies and violent thunderstorms. These light and variable winds are called the
'Doldrums'. Above and below the Doldrums, up to 30 ~ north and south, are the Trade
Winds, which blow with considerable steadiness, interrupted by occasional storms. Land
and sea breezes (mentioned later) also affect their behaviour.

Above and below 30 ~ of latitude, as far as the sub-polar regions, the Westerlies blow.
They are the result of the three factors mentioned earlier, but their behaviour is very much
influenced by the development of regions of low pressure, termed cyclones, producing
storms of the pattern familiar in temperate zones. This cyclonic influence means that the
weather is much less predictable in the temperate zones, unless the place in question is in
a very large land mass--for example in Asia or in North America. In temperate areas
consisting of a mixture of islands, broken coastline and sea, as in north-western Europe,
cyclonic weather is the rule and long-term behaviour difficult to forecast. The whole
matter is much complicated by the influence of warm and cold currents of water.

\section{Mist and fog}
For condensation to occur in the atmosphere the presence is required of small, solid
particles termed condensation nuclei. Any small solid particle will not do; it is desirable
that the particles should have some affinity for water. Hygroscopic materials such as salt
and sulphur dioxide then, play some part in the formation of condensation. The present
opinion appears to be that the products of combustion play an important part in the provision
of condensation nuclei and that the size and number of these nuclei vary tremendously.

Over industrial areas there may be several million per cubic centimetre of air whereas, over
sea, the density may be as low as a few hundred per cubic centimetre.

These nuclei play an important part in the formation of rain as well as fog, but for fog
to form the cooling of moist air must also take place. There are two common sorts of fog:

advection fog, formed when a moist sea breeze blows inland over a cooler land surface,
and radiation fog. Radiation fog forms when moist air is cooled by contact with ground
which has chilled as the result of heat loss by radiation to an open sky. Cloud cover
discourages such heat loss and, inhibiting the fall of surface temperature, makes fog formation
less likely. Still air is also essential; any degree of wind usually dissipates fog fairly
rapidly. Fog has a tendency to occur in the vicinity of industrial areas owing to the local
atmosphere being rich in condensation nuclei. Under these circumstances, the absence of
wind is helpful in keeping up the concentration of such nuclei. The dispersion of the nuclei
is further impeded by the presence of a temperature inversion, that is, by a rise in air
temperature with increase of height, instead of the reverse. This discourages warm air from
rising and encourages the products of combustion and fog to persist, other factors being
helpful.

Long-wave thermal radiation, ILW, directed to the sky at night can be considered as
radiation to a black body at a temperature of absolute zero. This is modified by a correction
factor to account for variations in the absorption by water vapour in the lower reaches of
the atmosphere. This changes as the amount of cloud cover and the area of the sky seen by
the surface varies. According to Brunt (1932) the absorption effect can be accounted for by
a vapour correction factor, $K$, expressed by

\begin{equation}
K = 0.56 - 0.08 \sqrt{p_s}
\end{equation}

where Ps is the vapour pressure of the air in millibars. The amount of sky seen by a surface
is given by an angle factor, B,

\begin{equation}
B = 0.5(1 + \cos \delta)
\end{equation}

where $\delta$  is the acute angle between the surface and the horizontal. Thus for a flat roof 8 =
0 and B = 1, whereas for a wall 8 = 90 ~ and B = 0.5. Cloud cover can play a part as well
and if we assume it is wholly effective in suppressing loss from the surface to outer space,
a cloud cover factor, C, can be introduced. Table 5.1 gives typical, approximate, average
cloud cover factors for Kew, based on data recorded for the amount of bright sunshine
received between sunrise and sunset in relation to the maximum amount of sunshine that
could be received during the same period of the day.

\begin{minipage}{5in}
\begin{tabular}{lllllllllllll}
\toprule
Month &Jan &Feb &Mar &Apr &May &Jun &Jul &Aug &Sep &Oct &Nov &Dec\\
\midrule
C &0.2 &0.2   &0.3   &0.4   &0.4  &0.4  &0.4  &0.4  &0.4  &0.3 &0.2 &0.2\\
\bottomrule
\end{tabular}
\end{minipage}
\def\badcheck{A penalty has been added because your
check to us was not honored by your bank.\par}
\def\cheater{A penalty of 50\% of the underpaid tax
has been added for fraud.\par}

\chapter{Solar Heat gains}

\section{The composition of heat gains}

Heat gains are either sensible, tending to cause a rise in air temperature, or latent, causing
an increase in moisture content. In comfort air conditioning sensible gains originate from
the following sources:

(i) Solar radiation through windows, walls and roofs.

(ii) Transmission through the building envelope and by the natural infiltration of warmer
air from outside.

(iii) People.

(iv) Electric lighting.


Latent heat gains are due to the presence of the occupants and the natural infiltration of
more humid air from outside.

In the case of industrial air conditioning there may be additional sensible and latent heat
gains from the processes carried out.

All the above sources of heat gain are well researched but a measure of uncertainty is
introduced by the random nature of some, such as the varying presence of people and the
way in which electric lights are switched. The thermal inertia of the building structure also
introduces a problem when calculating the sensible heat gain arising from solar radiation.
It follows that a precise determination of heat gains is impossible. Nevertheless, it is vital
that the design engineer should be able to calculate the heat gains with some assurance and
this can be done when generally accepted methods of calculation are followed, supported
by sound common sense. The following text discusses and describes such methods.


\section{The physics of solar radiation}

The sun radiates energy as a black body having a surface temperature of about 6000°C
over a spectrum of wavelengths from 300 to 470 nm. Nine per cent of the energy is in the
ultra-violet region but 91 per cent of the energy is in the visible part of the spectrum (380-
780 nm) and in the infra-red. Figure 7.1 shows a typical spectral distribution of the energy
reaching the surface of the earth. The peak intensity of the solar energy reaching the upper
limits of the atmosphere of the earth is about 2200 W mV2 at 480 nm but the average total
termed the \emph{solar constant}, is 1367  $W\/  m^{-2}$ \sidenote{Use SI units for consistency}, according to Iqbal(1983). The orbit of the earth
about the sun is an ellipse and the earth is slightly closer to the sun in January than it is in
July. Consequently the solar irradiation\index{solar irradiation}  has a maximum value of 1413 W mP2 in January
and a minimum of 1332 W m-2 in July.

A total of only about 1025 W m-2 reaches the surface of the earth when the sun is
vertically overhead in a cloudless sky. Of this figure, about 945 W m-2  is by radiation
received directly from the sun, the remainder being solar radiation received indirectly from
the sky.