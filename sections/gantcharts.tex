\chapter{More complicated Gantt Charts}
\label{ganttcharts}
The Gantt Charts basic style has been developed by 
Martin Kumm and is posted at his website \url{http://martin-kumm.de/tex_gantt_package.php}. The package here has been been modified slightly to keep with the idea of simplicity and to automate some aspects of it. 


Gantt Charts are used extensively for Project Management, although they have received criticism from many circles, including Tufte who calls them \textit{analytically thin} and he is correct in many respects. If you do use them keep them simple and do not use background patterns, like a crosshatch or diagonal lines, instead of colors. They distract. Tufte writes that: 

\begin{quotation} 
   Background patterns in information graphics are evil.
\end{quotation} 

How could all those big projects in the last 3,000 years ever been done without a Gantt chart? Very complex things can be done without the aid of the central authority of the Gantt chart.

Most of the Gantt charts are analytically thin, too simple, and lack substantive detail. The charts should be more intense. At a minimum, the charts should be annotated--for example, with to-do lists at particular points on the grid. Costs might also be included in appropriate cells of the table. About half the charts show their thin data in heavy grid prisons. For these charts the main visual statement is the administrative grid prison, not the actual tasks contained by the grid. See the discussion at Tufte \url{http://www.edwardtufte.com/bboard/q-and-a-fetch-msg?msg_id=000076} 


\begin{fullwidth}
\newcommand{\gbar}[3]{\ganttbar[crosshatch, color=orange]{#1}{#2}{#3}}

\definecolor{lightgray}{rgb}{0.7,0.7,0.7}
%% #1 11 rows deep
%% #2 24 cells wide

%% Define short-cuts for months
%%


\def\gantscale#1{}

\newcounter{acount}
\setcounter{acount}{0}
\stepcounter{acount}
\addtocounter{acount}{10}



%\ifnum\theacount>10 number is greater than ten \else number is less than ten\fi


%% Define the months just in case they have not been set
%% before. This needs to be internationalized.
\def\mon#1{\ifcase#1\or%
  Jan\or Feb\or Mar\or Apr\or May\or Jun\or%
  Jul\or Aug\or Sept\or Oct\or Nov\or Dec\else Error 
\fi}


%% a helper macro to add a comma and a space to make the
%% code later on a bit more cleaner

\def\addcomma{, }

%% We define a new counter for looping through the months
%% in order to create the calendar on top
%% this will need to be expanded later 
%\newcounter{monthcount}
%\setcounter{monthcount}{1}
%\whiledo{\value{monthcount}<12}
%  {%
%   \mon{\themonthcount}%
   % adds comma except last one
 %  \ifnum\themonthcount<11 \addcomma\else\relax\fi %
   
  % \stepcounter{monthcount}% 
  %}



\mon{\numexpr 1+3\relax} \intcalcAdd{\number\year}{1} \relax

%% Global setup
%% 

\xdef\monthstoplot{48}
\xdef\linesdeep{13}

%% Chart drawn in the margins
%% This is the main routine for the chart

%% We define a scale
%
\def\ganttscale#1{#1}



\hskip-2.5cm\begin{gantt}[xunitlength=\ganttscale{0.25cm},  fontsize=\footnotesize, drawledgerline=false]{\linesdeep}{\monthstoplot}

%% Define the title of the chart
%%
%% 
\newcommand\GanttTitle[1]{
\ganttitle
    \titleelement{\textsc{#1}}{\monthstoplot}
\endganttitle
}

\GanttTitle{Test}

\begin{ganttitle}
%% We define a new counter for looping through the months
%% in order to create the calendar on top
%% this will need to be expanded later 
\newcounter{monthcount}
\setcounter{monthcount}{1}
\whiledo{\value{monthcount}<13}
  {%
   \mon{\themonthcount}%
   % adds comma except last one
   %\ifnum\themonthcount<11 \addcomma\else\relax\fi %
   \titleelement{\mon{\themonthcount}}{4}
   \stepcounter{monthcount}% 
  }
 
\end{ganttitle}

%% We now loop to typeset the week labels
%% this needs to become more sophisticated
%% as some months do not have 4 weeks
\begin{ganttitle}
   \newcounter{weekcount}
   \setcounter{weekcount}{1}
   \whiledo{\value{weekcount}<13}
  {%
   \mon{\theweekcount}%
   \numtitle{1}{1}{4}{1}
   \stepcounter{weekcount}% 
  }      
\end{ganttitle}
 
%% We define some new colors
\definecolor{red}{rgb}{1,0,0}
% activities are plotted here
\xdef\prop{crosshatch, color=orange}
\ganttbar[crosshatch, color=orange]{Physical completion   \protect{[\ref{g:test}}] }{0}{2}
\ganttbarcon[dots, color=orange]{Hydrotest}{2}{0.2}
\ganttbarcon[crosshatch, color=orange]{Snagging}{2.2}{0.5}
\ganttbar[crosshatch, color=orange]{Electrical}{2}{2}
\ganttbar[crosshatch, color=orange]{Podium Completion}{1}{4}
\ganttbar[crosshatch, color=orange]{Flushing}{1}{4}
\ganttbarcon[crosshatch, color= orange]{Qatar Cool Inspection}{3}{4}
\ganttbarcon[dots, color=orange]{Qatar Cool connection}{9}{1.3}
\ganttbarcon[dots, color=orange]{Operate}{12}{1.3}
\ganttbarcon[dots,color=red]{Handover}{14}{2}

\end{gantt}

\end{fullwidth}



   
\section{The gantt package}

In the following you will find a short description of environments and commands: 

The gantt environment draws the canvas of a gantt figure (realized as tikzpicture)
The usage is |begin{gantt}[...]{no of Tasks to plot}{no of time slots}|
The optional argument |[...]| can be filled in a |key=value| syntax, using one or more of the following keys:

\begin{description}
\item [xunitlength]  length of one time slot (default: 1 cm)
\item [fontsize] fontsize of labels (default: |\normalsize|)
\item [titlefontsize] fontsize of title section (default: |\small|)
\item [drawledgerline] Switch to enable/disable the drawing of horizontal ledger lines (default value: false)
\item [ganttitle] is the environment for drawing the title section
\item [titleelement] draws one element of the title
usage: |\titleelement{label}{length}}|
\end{description}




The \cmd{numtitle} draws a numbered sequence of title elements:
usage: 

\begin{teX}
\numtitle{start number}{increment}{end number}{length of each title element}
\end{teX}

\cmd{ganttbar} draws a single, unconnected bar for representing a task
usage: 

\begin{teX}
\ganttbar[pattern]{label}{start}{length}
\end{teX}

where the optional pattern argument is a tikz pattern, nice patterns for tasks are: 
north west lines (default), north east lines, crosshatch, crosshatch dots, grid, ...


\begin{verbatim}
\cmd{ganttcon} draws an arrow between to bars with specified coordinates
usage: \ganttbar{startx}{starty}{endx}{endy}

\ganttbarcon draws a single bar \textit{and} connects the bar with the previous bar for consecutive tasks
usage: 

\begin{teX}
\ganttbar[pattern]{label}{start}{length}
\end{teX}

where the optional pattern argument is a tikz pattern, nice patterns for tasks are: north west lines (default), north east lines, crosshatch, crosshatch dots, grid.

\ganttgroup draws a bar to group tasks
usage: \ganttgroup{label}{start}{length}

\end{verbatim}
\section{Physical completion}

\label{g:test}

\numberLineAt{10}
\begin{teX}
\def\mon#1{\ifcase#1\or%
  Jan\or Feb\or Mar\or Apr\or May\or Jun\or%
  Jul\or Aug\or Sept\or Oct\or Nov\or Dec\else Error 
\fi}
\end{teX}











These are notes for the gantt chart
